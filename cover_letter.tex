\documentclass[11pt]{letter}
\usepackage[utf8]{inputenc}
\usepackage[T1]{fontenc}
\usepackage{amsmath}
\usepackage{geometry}
\geometry{margin=1in}
\usepackage{hyperref}

\signature{Ian Todd\\Sydney Medical School\\University of Sydney}
\address{Sydney Medical School\\University of Sydney\\Sydney, NSW 2006, Australia\\itod2305@uni.sydney.edu.au}

\begin{document}

\begin{letter}{Editorial Office\\npj Health Systems}

\opening{Dear Editors,}

I am pleased to submit ``The Dimensional Validity Bound: A Structural Limit on Clinical AI in Multimorbidity'' for consideration at \textit{npj Health Systems}.

\textbf{Why this matters now.} The Epic Sepsis Model's external validation failure (AUC 0.63), IBM Watson Oncology's withdrawal, and clinical alert override rates exceeding 90\% share a common cause: patient complexity exceeding algorithmic tractability. The standard response---more data, better models---assumes these are technical limitations. We demonstrate they are structural impossibilities.

\textbf{What we show.} When patient state dimensionality exceeds observer capacity, no amount of additional training data can restore validity. We formalize this as the \textit{dimensional validity bound}: the critical ratio $r = D_{\text{observer}}/D_{\text{patient}}$ below which algorithmic coupling becomes unstable. Using MIMIC-IV (N=425,216 hospitalizations), we demonstrate:
\begin{itemize}
    \item Effective dimensionality \textit{decreases} with multimorbidity ($D_{\text{eff}}$ = 44.3 $\to$ 37.5 $\to$ 29.5), supporting the Lipsitz-Goldberger ``loss of complexity'' hypothesis---illness is a manifold collapse
    \item Classifier AUC follows a U-shaped pattern: 0.867 (low), \textbf{0.835} (moderate), 0.859 (high multimorbidity)---worst performance in the ``zone of maximum entropy'' where priors are flat and outcomes not yet deterministic
    \item Diagnostic cascade probability exceeds 50\% beyond 14 tests; coupling destabilizes at $r < 0.3$
\end{itemize}

\textbf{The conceptual advance.} Berisha et al.\ (2021, \textit{npj Digital Medicine}) identified the curse of dimensionality in clinical AI as a dataset problem. We extend this to its logical conclusion: even perfect datasets cannot overcome intrinsic dimensional mismatch between algorithms and complex patients. This reframes AI failure in multimorbidity from ``needs more development'' to ``operating outside valid bounds.''

\textbf{Clinical and regulatory implications.} We propose Dimensional Validity Audits as a regulatory requirement for clinical AI---analogous to subgroup efficacy requirements for pharmaceuticals. Complexity metrics should appear in Model Cards, trigger real-time confidence warnings, and inform postmarket surveillance. This speaks directly to FDA Good Machine Learning Practice, the EU AI Act, and emerging TGA/MHRA frameworks.

\textbf{Fit with npj Health Systems.} This work sits at the intersection of clinical AI limitations, health system implementation challenges, and actionable regulatory translation---core themes of the journal. The dimensional validity bound provides the conceptual vocabulary for a problem clinicians experience daily but lack formal language to articulate.

The manuscript has not been published elsewhere and is not under consideration at another journal. All authors (sole author) approve the submission.

\closing{Sincerely,}

\end{letter}
\end{document}
